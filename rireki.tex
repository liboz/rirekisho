\documentclass[a4paper]{article}
\usepackage{xeCJK}
\setCJKmainfont[BoldFont=NotoSansCJKjp-Bold,AutoFakeSlant=0.15,Path = fonts/noto/]{NotoSansCJKjp-Regular}
\setCJKmonofont[BoldFont=NotoSansMonoCJKjp-Bold,Path=fonts/noto/]{NotoSansMonoCJKjp-Regular}
\usepackage{rireki}
%
% オプション
%
% 下記のオプションが利用可能です。
%
% \空行挿入		% 学歴と職歴の間に空行を挿入します
\begin{document}
%
% ID情報
%
\姓{\Large 履歴}
\名{\Large 一朗}
\姓読み{りれき}
\名読み{いちろう}
\性別{男}					% 男|女
\誕生日{平成1年2月3日}
\年齢{35}
%
% 顔写真
%
% 画像ファイルにはEPS フォーマット・縦横比4:3 のものをご使用ください。
% 縦を4cm に調整し、縦横比を変更せずに印刷します。
% 次のように指定します。
% \顔写真{photo.eps}
%
\顔写真{}
%
% 現住所
%
\現住所郵便番号{123}
\現住所{〇〇市〇〇町 1--2--3}
\現住所読み{まるまるし まるまるちょう}
\現住所市外局番{0123}
\現住所電話番号{45-6789}
\現住所呼び出し{〇〇方}
%
% 連絡先
%
\連絡先郵便番号{}
\連絡先{\tt taro@network.or.jp}
\連絡先読み{}
\連絡先市外局番{}
\連絡先電話番号{1234-56-7890}
\連絡先呼び出し{}
%
% 学歴、職歴
%
% 学歴、職歴を年月順に列挙してください。合計20個まで記入出来ます。
% 20個を超える部分は印刷されませんので、ご注意ください。
% 印刷順は、学歴=>職歴の順になります。
%
\学歴{平成1}{4}{〇〇市立〇〇高等学校 入学}      % {年}{月}{内容}
\学歴{平成2}{3}{〇〇市立〇〇高等学校 卒業}
\学歴{平成3}{4}{〇〇大学 入学}
\学歴{平成4}{3}{〇〇大学 卒業}
\学歴{平成5}{4}{〇〇大学大学院 入学}
\学歴{平成6}{3}{〇〇大学大学院 修了}
\学歴{平成7}{4}{専門学校〇〇 入学}
\学歴{平成8}{3}{専門学校〇〇 卒業}
\職歴{平成9}{4}{株式会社〇〇 入社}
\職歴{平成10}{9}{株式会社〇〇 退職}
\職歴{平成11}{10}{株式会社〇〇 入社}
\職歴{平成12}{10}{株式会社〇〇 退職}
\職歴{平成13}{10}{有限会社〇〇 入社}
\職歴{平成14}{8}{有限会社〇〇 退職}
\学歴{平成15}{9}{〇〇〇大学 入学(海外留学)}
\学歴{平成16}{9}{〇〇〇大学 中退}
\職歴{平成17}{10}{株式会社〇〇入社}
\職歴{平成18}{10}{株式会社〇〇退職}
\職歴{平成18}{10}{現在無職}
%
% 資格
%
% 資格を取得年月順に列挙してください。9つまで記入できます。
% 9つを超える部分は印刷されませんので、ご注意ください。
%
\資格{平成1}{4}{普通自動車一種免許}            % {取得年}{取得月}{資格}
\資格{平成2}{9}{自動二輪免許}
\資格{平成3}{4}{第二種情報処理技術者}
\資格{平成4}{4}{第一種情報処理技術者}
\資格{平成5}{3}{宅地取り引き主任者}
%
% 個人情報
%
% 志望の動機と本人希望記入欄はlatex のコマンドを記述できます。
%
\志望の動機{
	\begin{tabular}{ll}
	{ 志望の動機} & 〇〇〇〇〇〇〇〇〇〇〇〇〇〇\\
	{ 特技} & 〇〇〇\\
	{ 好きな学科} & 〇〇〇\\
	{ アピールポイント} & 〇〇〇〇〇〇〇〇〇〇〇〇〇\\
	\end{tabular}
}

\本人希望記入欄{
	私が希望する仕事の条件は下記の通りです。
	\begin{itemize}
	\item 〇〇〇〇〇〇〇〇〇〇〇〇〇〇
	\item 〇〇〇〇〇〇〇〇〇〇〇〇〇〇
	\item 〇〇〇〇〇〇〇〇〇〇〇〇〇〇
	\end{itemize}
}

%
% その他
%
\通勤時間{約 1時間30分}
\扶養家族数{2}					% 人数(配偶者を除きます)
\配偶者{あり}					% あり|なし
\配偶者の扶養義務{なし}				% あり|なし

\サイン{Your Signature}

\end{document}
